\documentclass[11pt]{beamer}
\usepackage[english]{babel}    % faire du français
\usepackage[T1]{fontenc}        % accents dans le DVI
\usepackage[utf8]{inputenc}   % accents dans le source
\usepackage{graphicx}
\usepackage{amssymb}
\usepackage{mathtools}
\usepackage{color, colortbl}

\usepackage{libertine}
\renewcommand*\familydefault{\sfdefault}  %% Only if the base font of the document is to be sans serif

\setbeamersize{text margin left=10pt,text margin right=10pt}
\setbeamertemplate{navigation symbols}{}

\pgfdeclareimage[height=\textheight,width=\textwidth]{bkg}{frontAgaetis}
\setbeamertemplate{background}{\pgfuseimage{bkg}}

\setbeamercolor{frametitle}{fg=orange}
\setbeamercolor{itemize item}{fg=orange}
\setbeamertemplate{itemize item}[circle]
\setbeamercolor{itemize subitem}{fg=orange}
\setbeamertemplate{itemize subitem}[triangle]
\setbeamercolor{enumerate item}{fg=orange}
\setbeamercolor{enumerate subitem}{fg=orange}

\usefonttheme[onlymath]{serif}

\definecolor{paleOrange}{RGB}{255,204,153}
\definecolor{forestGreen}{RGB}{0,204,0}
\definecolor{veryPaleOrange}{RGB}{255,229,204}
\setbeamertemplate{blocks}[rounded][shadow=true]
\beamerboxesdeclarecolorscheme{suppervise}{orange}{paleOrange}

\newcommand{\tabitem}{~~\llap{\textbullet}~~}

\author{\textcolor{black}{Introduction à l'apprentissage automatique} \hfill \textcolor{orange}{Léo Beaucourt} \hfill \textcolor{black}{SIGMA - Agaetis}}
\setbeamertemplate{footline}
{                  
  \leavevmode%
  \hbox{\begin{beamercolorbox}[wd=.85\paperwidth,ht=2.5ex,dp=1.125ex,leftskip=.3cm plus1fill,rightskip=.3cm]{author in head/foot}%
      \insertauthor
  \end{beamercolorbox}}%
  \hbox{\begin{beamercolorbox}[wd=.05\paperwidth,ht=2.5ex,dp=1.125ex,leftskip=.3cm plus1fill,rightskip=.3cm]{author in head/foot}%
      \textcolor{orange}{\insertframenumber/\inserttotalframenumber}
  \end{beamercolorbox}}%
  \vskip0pt%
}

\begin{document}

\begin{frame}

  \vspace{5cm}
  \begin{center}
    \Large
    \textcolor{orange}{Introduction à l'apprentissage automatique}\\
    \normalsize
    \vspace{0.5cm}
    Léo Beaucourt
  \end{center}
    
\end{frame}

\setbeamertemplate{background}{}

\begin{frame}{Contenu du cours}
  \begin{enumerate}
  \item Intro: Qu'est ce que le \textit{Machine Learning}?
  \item L'apprentissage supervisé
    \begin{enumerate}
    \item La régression linaire
    \item La régression logistique
    \item Les problèmes de biais et de variances
    \item Les arbres de décisions
%    \item Apprentissage supervisé, les bonnes pratiques
    \end{enumerate}
  \item L'apprentissage non supervisé
    \begin{enumerate}
    \item Les algorithmes de clustering
    \item Analyse en composantes principales
    \end{enumerate}
  \item L'apprentissage profond
    \begin{enumerate}
    \item Les réseaux de neurones
    \item Les bonnes pratiques
    \item Les différentes architectures de NN
    \end{enumerate}
  \end{enumerate}
\end{frame}

\begin{frame}{À propos de ce cours}
  \begin{itemize}
  \item Introduction à l'apprentissage automatique (ou \textit{machine learning} en anglais)
  \item En pratique: \textit{Python}, \textit{Jupyter}. Packages \textit{numpy} \textit{pandas}, \textit{matplotlib}, \textit{scikit-learn} et \textit{tensorflow}.
  \item Pas de pré-requis mathématiques (à part les dérivées partielles, l'algébre linéaire, ...)
  \item Largement inspiré de l'excellent (et complet!) cours de \textbf{\textcolor{orange}{Andrew Ng}} sur \href{https://www.coursera.org/learn/machine-learning}{\color{blue}{Coursera}}. 
  \end{itemize}
  
  \vfill
  \begin{center}
    \large
    \textcolor{orange}{Allez, on démarre en douceur!}
  \end{center}
\end{frame}

\begin{frame}{1. Intro: Qu'est ce que le \textit{Machine Learning}}
  \begin{itemize}
  \item \textbf{Arthur Samuel:}
    \begin{itemize}
      \normalsize
    \item \textit{The field of study that gives computers the ability to learn without being explicitly programmed.}
    \end{itemize}
    \vspace{0.2cm}
  \item \textbf{Tom Mitchell:}
    \begin{itemize}
      \normalsize
    \item \textit{A computer program is said to learn from experience E with respect to some class of tasks T and performance measure P, if its performance at tasks in T, as measured by P, improves with experience E.}
    \end{itemize}
    \vspace{0.5cm}
  \item \textbf{\textcolor{orange}{L'idée: }} Une machine apprend \textit{seule} à réaliser une tache complexe à l'aide de processus itératifs simple.
  \end{itemize}
\end{frame}

\begin{frame}{1. Intro: Qu'est ce que le \textit{Machine Learning}}
  \begin{figure}
    \includegraphics[width=0.6\textwidth]{fig/aiVennDiagram.png}
  \end{figure}
  \tiny
  \vspace{-1cm}
  \textit{[From \href{http://www.deeplearningbook.org/}{\color{blue}{MIT Press book Deep Learning}}]}
\end{frame}

\begin{frame}{1. Intro: Qu'est ce que le \textit{Machine Learning}}
  \begin{figure}
    \includegraphics[width=0.6\textwidth]{fig/mlvsAI.png}
  \end{figure}
\end{frame}

\begin{frame}{1. Intro: Qu'est ce que le \textit{Machine Learning}}
  \begin{itemize}
  \item Les principaux types d'apprentissage:
  \end{itemize}
  \hspace{0.25\textwidth}
  \begin{beamerboxesrounded}[scheme=suppervise,width=0.5\textwidth]{\textcolor{black}{Supervisé}}
    \begin{itemize}
      \tiny
    \item Utilise des données \textit{labélisées}
    \item La machine apprend par l'exemple
    \item \textit{Prédis} le résultat pour de nouveaux événements
    \item Problèmes de régression et de classification
    \item Regression linéaire et logistique
    \item Réseaux de Neurones
    \item Arbres de décisions
    \end{itemize}
  \end{beamerboxesrounded}

  \vfill
  
  \begin{minipage}{.5\textwidth}
    \begin{beamerboxesrounded}[scheme=suppervise,width=0.95\textwidth]{\textcolor{black}{Non-supervisé}}
      \begin{itemize}
        \tiny
      \item Données non \textit{labélisées}
      \item La machine apprend par elle même à indentifier une structure
      \item Évaluation des performances compliqué.
      \item Problèmes de classification, réduction de dimensions
      \item K-means
      \item Analyse en Composante Principale
      \end{itemize}
    \end{beamerboxesrounded}
  \end{minipage}
  \hfill
  \begin{minipage}{.5\textwidth}
    \begin{beamerboxesrounded}[scheme=suppervise,width=0.95\textwidth]{\textcolor{black}{Par renforcement}}
      \vspace{0.5cm}
      \begin{itemize}
        \tiny
      \item Un agent A, effectue une action Ac, l'environnement E lui renvoie une récompense.
      \item Récompenses à court et long terme
      \item Utilisé par Deepmind (alphaGo)
      \end{itemize}
      \vspace{0.5cm}
    \end{beamerboxesrounded}
  \end{minipage}  
\end{frame}

\begin{frame}{2. L'apprentissage supervisé}
  \begin{itemize}
  \item Utilise des données \textit{labélisées}
  \item La machine apprend par l'exemple
  \item \textit{Prédis} le résultat pour de nouveaux événements
  \item Problèmes de régression et de classification
  \item Regression linéaire et logistique
  \item Réseaux de Neurones
  \item Arbres de décisions
  \end{itemize}
\end{frame}

\begin{frame}{2.1 La regression linéaire}
  \begin{itemize}
    
  \item Déterminer une relation \textit{linéaire} entre \textit{input(s)} (features) et \textit{output}:
    \begin{center}
      \normalsize
      \boldmath $\Rightarrow$ \unboldmath \textbf{\textcolor{orange}{Apprentissage Supervisé}}
    \end{center}
  \item Prédiction d'une valeur \textbf{continue} (e.g. non discrète, non catégorielle)
  \item Applications:
    \begin{itemize}
      \normalsize
    \item Recherche de corrélations
    \item En science, modélisation de phénomènes (physiques, biologiques, ...)
    \item Dans le domaine médical: les études épidémiologique
    \item Dans la finance/économie: prédictions des tendances
    \item $\dots$
    \end{itemize}
  \end{itemize}
  \begin{center}
    \textbf{Sujet Data Science} \boldmath $\Rightarrow$ \unboldmath \textbf{Premier algorithme à tester!}
  \end{center}
\end{frame}

\begin{frame}{2.1 Un exemple: le prix d'une carte graphique}
  \begin{itemize}
  \item La propriété principale d'une carte Graphique: valeur de \textbf{GPU}
    \vspace{0.2cm}
  \item Données, liste de carte graphiques dont on connait le couple $\{GPU;prix\}$:
  \end{itemize}

  \begin{figure}
    \includegraphics[width=0.6\textwidth]{fig/gpuPrices.png}    
  \end{figure}
\end{frame}

\begin{frame}{2.1 Construire un modèle (regression linéaire)}
  \begin{itemize}
  \item Soit: $x_{1}$ la valeur de GPU de nos $m$ carte graphiques, et $y$ leur prix
    \vspace{0.2cm}
  \item On cherche à déterminer le modèle pour prédire un prix $\hat{y}$ à partir $x_{1}$:
    \begin{equation*}
      \hat{y} = h_{\theta}(x_{1})
    \end{equation*}
  \item On défini le paramètre $\theta_{1}$ qui va \textit{lier} $x_{1}$ à $\hat{y}$:
    \begin{equation*}
      h_{\theta}(x) = \theta_{1} x_{1}
    \end{equation*}
  \item Rappel math: \textbf{fonction linéaire} $f(x) = kx$
  \end{itemize}
\end{frame}

\begin{frame}{2.1 Construire un modèle (regression linéaire)}
  \begin{itemize}
  \item Initialisons aléatoirement la valeur de $\theta_{1}$
  \end{itemize}
  \vspace{-0.5cm}
  \begin{figure}
    \includegraphics[width=0.6\textwidth]{fig/model.png}
  \end{figure}
  \vspace{-0.5cm}
  \begin{itemize}
  \item C'est pas encore ça ...
  \end{itemize}
\end{frame}

\begin{frame}{2.1 La fonction de coût}
  \begin{itemize}
  \item  $J(\theta)$: \textit{véracité} de notre modèle
  \item Ex, somme quadratique des erreurs: $J(\theta) = \frac{1}{2m} \displaystyle\sum_{i=0}^{m}(\hat{y}^{(i)} - y^{(i)})^{2}$
  \end{itemize}
  \begin{figure}
    \includegraphics[width=0.6\textwidth]{fig/modelEstimation.png}
  \end{figure}  
\end{frame}

\begin{frame}{2.1 La fonction de coût}
  \begin{itemize}
  \item On cherche à trouver la valeur de $\theta_{1}$ qui \textbf{minimise} $J(\theta)$
    \vspace{0.2cm}
  \item En Brute ...
  \end{itemize}
  \vspace{-0.5cm}
  \begin{figure}
    \includegraphics[width=0.6\textwidth]{fig/costFct.png}
  \end{figure}
  \vspace{-0.5cm}
  \begin{itemize}
  \item ... essayons d'optimiser
  \end{itemize}
\end{frame}

\begin{frame}{2.1 La descente de gradient}
  \begin{itemize}
  \item Algorithme pour arriver ``\textit{rapidement}'' au minimum de $J(\theta)$ 
    \vspace{0.2cm}
  \item On va utiliser la \textit{dérivation}: $\frac{d}{d\theta_{1}}J(\theta)$:
  \end{itemize}
  \begin{center}
    Si $J(\theta)$ est croissant: $\frac{d}{d\theta_{1}}J(\theta) > 0$, \hspace{0.5cm}
    Si $J(\theta)$ est décroissant: $\frac{d}{d\theta_{1}}J(\theta) < 0$
  \end{center}
  \vspace{-0.5cm}
  \begin{figure}
    \includegraphics[width=0.5\textwidth]{fig/derivation.png}
  \end{figure}
  \vspace{-0.5cm}
\end{frame}

  
\begin{frame}{2.1 La descente de gradient}
  \begin{itemize}
  \item (Encore) un peu de math, la descente de gradient s'écrit:
  \end{itemize}
  \begin{beamerboxesrounded}[scheme=suppervise,width=\textwidth]{\textcolor{black}{Descente de gradient}}    
    \vspace{-0.2cm}
    \begin{equation*}
      \begin{matrix} \text{Répéter jusqu'à convergence:} & \{ & \\ & & \theta_{1} := \theta_{1} - \alpha \frac{d}{d\theta_{1}}J(\theta) \\ & \} & \end{matrix}
    \end{equation*}
    \vspace{-0.2cm}
    \end{beamerboxesrounded}
    \begin{itemize}
    \item $\alpha$: taux d'apprentissage (\textit{learning rate}), \textbf{seul} paramètre de l'algorithme.
      \vspace{0.2cm}
    \item On va itérativement modifier la valeur de $\theta_{1}$ en fonction de la dérivée de $J(\theta)$, jusqu'à minimiser $J(\theta)$ (\textit{convergence}).
    \end{itemize}
\end{frame}

\begin{frame}{2.1 La descente de gradient}
  \begin{itemize}
  \item Dérivons donc notre fonction de coût:
  \end{itemize}
  \begin{equation*}
    J(\theta) = \frac{1}{2m} \displaystyle\sum_{i=0}^{m}(\hat{y}^{(i)} - y^{(i)})^{2} = \frac{1}{2m} \displaystyle\sum_{i=0}^{m}(\theta_{1}x_{1}^{(i)} - y^{(i)})^{2}
  \end{equation*}
  \begin{equation*}
      \frac{d}{d\theta_{1}}J(\theta) = \frac{1}{m}\displaystyle\sum_{i=0}^{m}(\hat{y}^{(i)} - y^{(i)}) x_{1}^{(i)}
  \end{equation*}
  \begin{itemize}
  \item Un peu de \textit{hand-tunning}:
    \begin{itemize}
      \normalsize
    \item Le learning rate ($\alpha$) est fixé à $0.03$ ($0.045$ pour la démo)
    \item Précision $\epsilon = 0.0001$ pour arrêter la descente de gradient 
    \end{itemize}
  \end{itemize}
\end{frame}

\begin{frame}{2.1 Préparation du dataset}
  \begin{itemize}
  \item Bonne pratique de ML, pour tout les algos!
  \item On sépare \textbf{aléatoirement} les données en 2 (3) échantillons:
    \begin{itemize}
      \normalsize
    \item Entraînement / (Validation) / Test
    \item \textit{80 / 20} (\textit{70 / 30}) ou \textit{60 / 20 / 20}
    \end{itemize}
    \vspace{0.2cm}
  \item \textit{Entraînement:} utilisé pour la descente de gradient
  \item \textit{Validation:} utilisé pour l'hyperparamètrage de l'algo
  \item \textit{Test:} utilisé pour mesurer la performance du modèle
  \end{itemize}
\end{frame}

\begin{frame}{2.1 C'est parti !}
  \begin{figure}
    \includegraphics[width=\textwidth]{fig/gradDescent.png}
  \end{figure}
  \begin{itemize}
  \item La descente de gradient c'est achevée au bout de quelques itérations
  \item On peut voir que $J(\theta)$ a continuellement diminué à chaque itération
  \end{itemize}
  \begin{center}
    $\theta_{1} ~ \approx ~ 80$ \\
    $err_{train} ~ \approx ~ err_{test}$
  \end{center}
\end{frame}

\begin{frame}{2.1 On peut maintenant faire une prédiction}
  \begin{itemize}
  \item Quel serait le prix de cartes avec 5, 10 et 14 Go de GPU? 
  \end{itemize}
  \vspace{-0.2cm}
  \begin{figure}
    \includegraphics[width=0.6\textwidth]{fig/pred.png}
  \end{figure}
  \vspace{-0.5cm}
  \begin{itemize}
  \item On pourra les vendre autour de 400, 800 et 1100 euros!
  \end{itemize} 
\end{frame}

\begin{frame}{2.1 Le choix du taux d'apprentissage}
  \begin{itemize}
  \item \boldmath $\alpha$ \textbf{Trop grand:} la descente de gradient diverge
  \item \boldmath $\alpha$ \textbf{Trop petit:} la descente de gradient est très longue
  \end{itemize}
  \begin{figure}
    \includegraphics[width=0.9\textwidth]{fig/learningRateChoice.png}
  \end{figure}
\end{frame}

\begin{frame}{2.1 Un mot sur la regression linéaire multivariables}
  \begin{itemize}
  \item Même principe, mais avec plusieurs variables $x_{i}$ (donc plusieurs paramètres $\theta_{i}$)
  \item on peut rajouter un biais $\theta_{0}$, cad un terme constant: $\hat{y} = \theta_{0} + \theta_{1}x_{1} + \dots$
  \item Notre fonction hypothèse s'écrie alors:
  \end{itemize}
  \begin{equation*}
    h_{\theta}(x) = \theta_{0} + \theta_{1}x_{1} + \theta_{2}x_{2} + \dots + \theta_{n}x_{n} = \theta_{0} + \displaystyle\sum_{i=1}^{n} \theta_{i} x_{i}
  \end{equation*}
  \begin{itemize}
    \item \textbf{\textcolor{orange}{Astuce:}} On définit $x_{0} = 1$, et on re-écrit la fonction hypothèse:
  \end{itemize}
  \begin{equation*}
    h_{\theta}(x) = \theta_{0}x_{0} + \theta_{1}x_{1} \dots + \theta_{n}x_{n} = \displaystyle\sum_{i=0}^{n} \theta_{i} x_{i}
  \end{equation*}

  \begin{itemize}
  \item La fonction de coût reste inchangée
  \end{itemize}
\end{frame}

\begin{frame}{2.1 Un mot sur la regression linéaire multivariables}
  \begin{itemize}
  \item Dans le cas multivariables, la descente de gradient devient:
  \end{itemize}
    \begin{beamerboxesrounded}[scheme=suppervise,width=\textwidth]{\textcolor{black}{Descente de gradient (cas multivariables)}}    
    \vspace{-0.2cm}
    \begin{equation*}
      \begin{matrix} \text{Répéter jusqu'à convergence:} & \{ & \\
        & & \theta_{0} := \theta_{0} - \alpha \frac{d}{d\theta_{0}}J(\theta) \\
        & & \theta_{1} := \theta_{1} - \alpha \frac{d}{d\theta_{1}}J(\theta) \\
        & & \theta_{2} := \theta_{2} - \alpha \frac{d}{d\theta_{2}}J(\theta) \\
        & & \dots \\
        & \} & \end{matrix}
    \end{equation*}
    \vspace{-0.2cm}
  \end{beamerboxesrounded}
    \begin{itemize}
    \item Il est très important de simultanément changer les valeurs des paramètres.
    \end{itemize}
\end{frame}

\begin{frame}{2.1 Un mot sur la regression linéaire multivariables}
  \begin{itemize}
  \item Pour illustrer: régression linéaire à deux dimensions
  \end{itemize}
  \begin{figure}
    \includegraphics[width=0.4\textwidth]{fig/multiVarData.png}
    \includegraphics[width=0.4\textwidth]{fig/multiVarCostFct.png}\\
    \includegraphics[width=0.45\textwidth]{fig/multiVarDesc.png}
  \end{figure}
\end{frame}

\begin{frame}{2.1 Affinons notre modèle de carte graphiques}
  \begin{itemize}
  \item Plus de features: chipset, fréquence, consommation, ...
  \item Il va falloir explorer et nettoyer les données:
    \begin{itemize}
      \normalsize
    \item Gestion des données manquantes / abbérantes
    \item \textit{Features engineering}
    \item Normaliser le dataset (pour accélérer la descente de gradient)
    \end{itemize}
  \end{itemize}
\end{frame}

\begin{frame}{2.1 Régression linéaire multivariables: Résultats}
  \begin{itemize}
  \item On utilise la même valeur de $\epsilon = 0.0001$ et $\alpha = 0.03$
  \item Plus long! Mais meilleur résultats:
  \item Modèle simple: $err ~ \approx ~ 100$
  \item Modèle multivariable: $err ~ \approx ~ 30$
  \end{itemize}
  \begin{figure}
    \includegraphics[width=0.45\textwidth]{fig/multiVarDesc.png}\\
  \end{figure}
\end{frame}
  
\begin{frame}{2.1 Pour conclure sur la régression linéaire}
  \begin{itemize}
  \item \textbf{Regression Linéaire:} $\hat{y}$ est une valeur \textit{continue}
    \begin{itemize}
      \normalsize
    \item Valeur discrète: \textbf{Regression Logistique} (\textit{classification})
    \end{itemize}
    \vspace{0.2cm}
  \item Le résultat $\hat{y}$ dépend \textbf{linéairement} des variables $x_{i}$ si:
    \begin{equation*}
      \hat{y} = \theta_{1}x_{1} + \dots + \theta_{n}x_{n} = \displaystyle\sum_{i=1}^{n} \theta_{i} x_{i}
    \end{equation*}
  \item \textbf{Supervisé}: $y$ connu pour chaque élément du le jeu de données d'entrainement
  \item Facile à implémenter (encore plus avec Scikit-learn ...), rapide: 
  \end{itemize}
  \begin{center}
    $\Rightarrow$ bon point de départ sur un sujet
  \end{center}
\end{frame}

\begin{frame}{2.1 Features scaling}
  \begin{itemize}
  \item Mettre les variables à la même échelle (performances)
  \item Normaliser les variables: $-1 \leq x_{i} \leq 1$
  \end{itemize}
  \vspace{-0.2cm}
  \begin{table}
    \footnotesize
        {\def\arraystretch{2}\tabcolsep=8pt
          \begin{tabular}{l|l|l}
            & \textbf{Feature Scaling} & \textbf{Mean normalization}\\
            \hline
            \textbf{Start range} & $x_{min} \leq x \leq x_{max}$ & $x_{min} \leq x \leq x_{max}$ \\
            \textbf{Transformation} & $x := \frac{x - x_{min}}{x_{max} - x_{min}}$ & $x := x - x_{mean}$ \\
            \textbf{New range} & $0 \leq x \leq 1$ & $(x_{min}-x_{mean}) \leq x \leq (x_{max}-x_{mean})$
          \end{tabular}
        }
  \end{table}
  \vspace{-0.2cm}
  \begin{itemize}
  \item En combinant les deux: \boldmath \textcolor{orange}{$x := \frac{x - x_{mean}}{x_{max} - x_{min}}$} $\Rightarrow$ \textcolor{orange}{$-1 \leq x \leq 1$} 
    \vspace{0.2cm}
    \footnotesize
  \item\textbf{Remarque:} Il est possible de remplacer $x_{max} - x_{min}$ par l'écart type: $\sigma_{x} = \sqrt{\frac{1}{n}\displaystyle\sum_{i=1}^{n}(x_{i} - \bar{x})^{2}}$
  \end{itemize}
\end{frame}

\begin{frame}{2.2 La régression logistique}
  \begin{itemize}
  \item \textbf{\textcolor{orange}{Classification}}: prédire un nombre limités de valeurs discrètes
    \begin{itemize}
    \item \textbf{Classification binaire}: Deux valeurs possibles: Vrai ou Faux (spam / non spam)
    \item \textbf{Classification multiclasse}: plusieurs valeurs possibles (camion, voiture, piétons, vélos, ...)
    \end{itemize}
    \vspace{0.5cm}
  \item \textcolor{orange}{\textbf{Classification binaire:}} En utilisant la régression linaire?
    \begin{itemize}
    \item $y \in \{0,1\}$ \textcolor{orange}{$\Rightarrow$} On défini un seuil $S$ pour $h_{\theta}(x)$:
      \begin{equation*}
        \begin{matrix*}[l]
          h_{\theta}(x) \geq S \rightarrow y = 1\\
          h_{\theta}(x) < S \rightarrow y = 0
        \end{matrix*}
      \end{equation*}
    \end{itemize}
  \item \textbf{Problème:} On voudrait que $0 \leq h_{\theta}(x) \leq 1$
  \end{itemize}
  \begin{center}
    $\Rightarrow$ Il faut redéfinir notre fonction hypothèse!
  \end{center}
\end{frame}

\begin{frame}{2.2 La régression logistique}
  \begin{itemize}
  \item Modèle de la régression logistique:
    \begin{itemize}
      \item On utilise la \textbf{Fonction Sigmoïde} $g(z)=\frac{1}{1+e^{-z}}$
    \end{itemize}
  \end{itemize}
  \begin{minipage}{0.4\textwidth}
    \begin{equation*}
      \begin{matrix*}[l]
        z = \displaystyle\sum_{i=0}^{n} \theta_{i} x_{i}\\
        h_\theta(x) = g(z) \\
      \end{matrix*}
    \end{equation*}
    \vfill
    \begin{equation*}
      \textcolor{blue}{0 \leq h_{\theta}(x) \leq 1}
    \end{equation*}
  \end{minipage}
  \begin{minipage}{0.5\textwidth}
    \begin{figure}
      \includegraphics[width=0.9\textwidth]{fig/logisticFct.png}
    \end{figure}
    \begin{center}
      \tiny
      \vspace{-0.5cm}
      \textcolor{blue}{$g(z)$} depuis \href{https://en.wikipedia.org/wiki/Sigmoid_function}{\color{blue}{Wikipedia}}
    \end{center}
  \end{minipage}
  \vfill
  \begin{itemize}
  \item \textbf{Classification:} On prédit une \textbf{probabilité}
  \end{itemize}
\end{frame}

\begin{frame}{2.2 La régression logistique}
  \begin{itemize}
  \item On change également la fonction de coût:
  \end{itemize}
  \begin{equation*}
    J(\theta)=\frac{1}{m} \displaystyle \sum_{i=1}^{m}Cost(h_{\theta}(x^{(i)}),y^{(i)})
  \end{equation*}
  \begin{itemize}
  \item Avec:
  \end{itemize}
  \begin{equation*}
    Cost(h_{\theta}(x),y) = 
    \begin{cases}
      -log(h_{\theta}(x)) & \quad \text{if } y = 1 \\
      -log(1 - h_{\theta}(x)) & \quad \text{if } y = 0 \\
    \end{cases}
  \end{equation*}
  \begin{figure}
    \includegraphics[width=.22\textwidth]{fig/logisticRegressionCostFunction1.png}
    \includegraphics[width=.2\textwidth]{fig/logisticRegressionCostFunction2.png}
  \end{figure}
\end{frame}

\begin{frame}{2.2 La régression logistique}
  \begin{itemize}
  \item Et la \textcolor{orange}{\textbf{Classification multiclasse}} ? $y \in \{0,1,2,\dots,n\}$
  \item La \textbf{descente de gradient} ne permet pas de la résoudre
  \item On peut \textit{tricher} en utilisant la méthode \textit{'One-vs-all'}:
    \begin{itemize}
    \item On remplace le problème multiclasse par $n+1$ problèmes binaire
    \item Probabilité que $y$ soit dans une classe ou qu'il soit dans une des autres:
    \end{itemize}
  \end{itemize}
  \begin{equation*}
    \begin{matrix*}[l]
      h_{\theta}^{(0)} = P(y=0 | x;0)\\
      h_{\theta}^{(1)} = P(y=1 | x;0)\\
      \dots \\
      h_{\theta}^{(n)} = P(y=n | x;0)\\
      \text{prediction} = max_{i}(h_{\theta}^{(i)}(x))\\
    \end{matrix*}
  \end{equation*}
  \begin{itemize}
  \item Algorithme d'optimisation avancés: \textit{Conjugate gradient}, \textit{(L-)BFGS}, $\dots$
  \end{itemize}
\end{frame}

\begin{frame}{2.3 Les problèmes de biais et de variance}
  \begin{figure}
    \includegraphics[width=0.9\textwidth]{fig/theProblemOfOverfitting.png}
  \end{figure}
  \footnotesize
  \vspace{-1cm}
  \begin{center}
    \textit{Underfitting} (Bias) \hspace{1.5cm} \textit{Just fine!} \hspace{1.5cm} \textit{Overfitting} (Variance)
  \end{center}
  \begin{itemize}
  \item \textbf{Underfitting}: modèle trop simple (pas assez de variables)
  \item \textbf{Overfitting}: modèle trop complexe, deux manière de résoudre:
    \begin{itemize}
    \item Réduire le nombres de variables ou changer les paramètres du modèle
    \item Utiliser des méthodes de \textbf{régularisation}
    \end{itemize}
  \end{itemize}  
\end{frame}

\begin{frame}{2.3 La régularisation}
  \begin{itemize}
  \item Contraindre les paramètres $\theta_{j}$ sans réduire le nombre de variables
  \item On ré-écrit la fonction de coût avec le \textcolor{blue}{terme de régularisation}:
  \end{itemize}
  \begin{equation*}
    J(\theta) = \frac{1}{2m}(\displaystyle\sum_{i=1}^{m}(h_{\theta}(x^{(i)}) - y^{(i)})^{2} + \textcolor{blue}{\lambda \displaystyle\sum_{j=1}^{n}\theta_{j}^{2}})
  \end{equation*}
  \begin{itemize}
  \item \boldmath $\lambda$: Paramètre de régularisation
  \item Si $\lambda$ est trop grand: risque d'\textit{underfitting}
  \item Si $\lambda = 0$: pas de régularisation. 
  \item \textbf{Remarque:} On ne régularise pas le terme constant $\theta_{0}$
  \end{itemize}
\end{frame}

\begin{frame}{2.4 Les arbres de décisions}
  \begin{itemize}
  \item Effectue une prédiction (regression ou classification) par une succession de decision simples: \textit{if-else-then} sur les différentes variables
  \item \textit{Arbre} = suite de décisions (\textit{noeuds}) ammenant à une prédiction (\textit{feuille})
  \end{itemize}
  \begin{minipage}{.45\textwidth}
    \begin{itemize}
    \item La complexité de la structure de données qu'il est possible de représenter avec une arbre de décision est proportionelle à la profondeur de l'arbre
    \end{itemize}
  \end{minipage}
  \begin{minipage}{.54\textwidth}
    \begin{figure}
      \includegraphics[width=.9\textwidth]{fig/decisionTreeExample.png}
    \end{figure}
    \begin{center}
      \scriptsize
      \textit{Un arbre pour déterminer la probabilité de survie des passagers du Titanic (depuis \href{https://en.wikipedia.org/wiki/Decision_tree_learning}{\color{blue}{Wikipedia}})}
    \end{center}
  \end{minipage}
\end{frame}

\begin{frame}{2.4 Les arbres de décisions: Apprentissage}
  \begin{itemize}
  \item La \textit{meilleure} séparation possible est déterminée:
    \begin{itemize}
      \normalsize
    \item Toutes les variables/sélection possibles
    \item Celle qui minimise une fonction de coût
    \end{itemize}
  \item L'échantillon d'entrainement est séparé suivant cette décision, créant ainsi deux feuilles (sous-échantillon)
  \item On recommence l'opération tant que l'on a pas atteint la profondeur voulue ou bien que la pureté de chaque feuille atteint un niveau satisfaisant (à déterminer)
  \end{itemize}
  \vspace{1cm}
  \textbf{Remarque:} Il est possible d'utiliser la même variables pour différents noeud.
\end{frame}

\begin{frame}{2.4 Les arbres de décisions}
  \begin{itemize}
  \item Ces algorithmes sont faciles à interpréter (et à visualiser)
  \item Tout les types de données (catégorielle, numérique, ...) peuvent être mélangés
  \end{itemize}
  \begin{figure}
    \includegraphics[trim={0 0 0 40},clip,width=.3\textwidth]{fig/spiralDTresult.png}
    \includegraphics[trim={0 0 0 40},clip,width=.3\textwidth]{fig/goodPartitionDT.png}
    \includegraphics[trim={0 0 0 40},clip,width=.3\textwidth]{fig/overfittingPartitionDT.png}
  \end{figure}
  \vspace{-1cm}
  \begin{center}
    \scriptsize
    \textit{Illustration depuis \href{https://www.classes.cs.uchicago.edu/archive/2013/winter/12200-1/assignments/pa4/index.html}{\color{blue}{University of Chicago}}}
  \end{center}
  \begin{itemize}
  \item Risque d'\textit{overfitting} (sur-apprentissage) du modèle qui se généralise mal
  \item Algorithmes très sensibles au jeu de données d'entrainement
  \end{itemize}
\end{frame}

\begin{frame}{2.4 Les arbres de décisions: méthodes d'ensemble}
  \begin{itemize}
  \item Pour pallier les faiblesse des arbres de décision, on les utilise comme \textit{base learners} dans des méthodes qui en regroupe plusieurs
  \item Il existe deux familles de méthodes:
    \begin{itemize}
      \normalsize
      \vspace{0.5cm}
    \item \textbf{averaging}: plusieurs estimateurs (\textit{learners}) indépendant dont on moyenne les prédictions (\textit{Random Forests}, $\dots$)
      \vspace{0.5cm}
    \item \textbf{boosting}: plusieurs estimateurs combinés (\textit{AdaBoost}, \textit{XGBoost}, $\dots$)
    \end{itemize}
  \end{itemize}
\end{frame}

\begin{frame}{2.4 Exemple d'algorithme d'\textit{averaging}: \textbf{Random Forests}}
  \begin{itemize}
  \item \textit{Bagging}: Chaque estimateur (arbre) de l'ensemble est construit à partir d'un sous-échantillon aléatoire (avec replacement) de l'échantillon d'entrainement (\textit{Bootstrap aggregating})
  \item \textit{Features bagging}: La \textit{meilleure} séparation possible sur un sous-échantillon aléatoire des variables
  \item Soit $f_{b}$, l'arbre entrainé sur le sous-échantillons $b$ ($b \in[1,B]$), on calcule la prédiction globale en moyennant les prédictions de tout les arbres:
  \end{itemize}
  \begin{equation*}
    \hat{f} = \frac{1}{B}\displaystyle\sum_{b=1}^{B} f_{b}(x)
  \end{equation*}
  \begin{itemize}
  \item Cette méthode obtient de meilleure performance car elle réduit la \textit{variance} du modèle sans accroitre le \textit{biais}
  \end{itemize}
\end{frame}

\begin{frame}{2.4 Exemple d'algorithme de \textit{boosting}: \textbf{AdaBoost}}
  \begin{itemize}
  \item L'apprentissage d'un même \textit{learner} est répété en modifiant le jeu de données à chaque itération
  \item On utilise des poids: $w_{i}$ pour $i \in [1,N]$. Initialement: $w_{i} = 1/N$
  \item À chaque itération, le poids des exemples correctement prédits diminue tandis que celui des exemples dont les prédictions sont fausse augmente
  \item L'algorithme devient plus sensibles aux exemples difficiles à prédire
  \end{itemize}
\end{frame}

%\begin{frame}{2.4 Exemple d'algorithme de \textit{boosting}: \textbf{XGBoost}}
%  \begin{itemize}
%  \item Obtient d'excellent résultats sur la plupart des cas d'usages (classification et régression)
%  \item Des arbres sont générés aléatoirement comme pour les \textit{random forests}
%  \item Mais au lieu de moyenner les prédictions, on additionne les prédictions
%    \begin{itemize}
%    \item À chaque étape, des arbres sont générés et on sélectionne celui qui optimise la fonction d'objectif (une fonction de coût + une fonction qui mesure la complexité du modèle)
%    \item On additionne la prédiction de cet arbre à la prédiction du modèle et on recommence jusqu'à atteindre une performance suffisante
%    \end{itemize}
%  \end{itemize}
%  \begin{equation*}
%    \begin{matrix*}[l]
%      \hat{y_{i}}^{(0)} = 0 \\
%      \hat{y_{i}}^{(1)} = \hat{y_{i}}^{(0)} + f_{1}(x_{i}) = f_{1}(x_{i})\\
%      \hat{y_{i}}^{(2)} = \hat{y_{i}}^{(1)} + f_{2}(x_{i}) = f_{1}(x_{i}) + f_{2}(x_{i})\\
%      \dots\\
%      \hat{y_{i}}^{(t)} = \hat{y_{i}}^{(t-1)} + f_{t}(x_{i}) = \displaystyle\sum_{k=1}^{t} f_{k}(x_{i}) \\
%    \end{matrix*}
%  \end{equation*}
%\end{frame}

\begin{frame}{2.4 Arbres de décision: \textit{Titanic}}
  \begin{figure}
    \includegraphics[width=\textwidth]{notebook/decisionTrees/titanic.pdf}
  \end{figure}
  \begin{table}
    \footnotesize
    \begin{tabular}{l|l|l|l|l}
      & \textbf{Decision Tree}  & \textbf{Random Forest}  & \textbf{AdaBoost}  & \textbf{XGBoost} \\
      \hline
      \textbf{Accuracy (\boldmath $\%$)} & 77.6 & 81.1 & 83.2 & 83.9 \\
    \end{tabular}
  \end{table}
\end{frame}


\begin{frame}{3 L'apprentissage non-suppervisé}
  \begin{itemize}
  \item Données non \textit{labélisées}
  \item La machine apprend par elle même à indentifier une structure
  \item Évaluation des performances compliqué.
  \item Problèmes de classification, réduction de dimensions
  \item \textit{K-means}, \textit{Mean Shift}, \textit{Gaussian Mixture Model}
  \item Analyse en Composante Principale
  \end{itemize}  
\end{frame}

\begin{frame}{3.1 Les algorithmes de clustering}
  \begin{itemize}
  \item \textit{Clustering}: Se rapproche d'un problème de classification (sans labels)
  \item Ces algorithmes cherchent à rassembler les exemples en cluster
  \item À la différence des arbres de décision, le choix du cluster ne s'effectue pas par une suite de décisions simple, mais en déterminant la plus petite distance possible dans l'espace des variables
  \item Utilisés pour la segmentation d'utilisateurs/marchés dans le commerce en ligne mais aussi en génétique
  \item Il faut (dans la majorité des cas) définir au préalable le nombre de clusters à construire (hyperparamètre du modèle)
  \end{itemize}
\end{frame}

\begin{frame}{K-Means: description et limites}
  \begin{itemize}
  \item Sépare les données en $K$ clusters $C_{k}$ d'égale variance (dispersion)
  \item L'algorithme modifie la position des \textit{centroïdes} $\mu_{k}$ afin de minimiser l'écart moyen de l'ensemble des points d'un cluster à son \textit{centroïde}
  \end{itemize}
  \vspace{-0.5cm}
  \begin{minipage}{0.55\textwidth}
    \begin{itemize}
    \item Critère de minimisation, \textbf{Inertie:}
    \end{itemize}
    \begin{equation*}
      I = \displaystyle\sum_{k=0}^{K}\displaystyle\sum_{x \in C_{k}}\|x-\mu_{k}\|^{2}
    \end{equation*}
    \begin{center}
      \textbf{\textcolor{orange}{(Distance Euclidienne)}}
    \end{center}
  \end{minipage}
  \hfill
  \begin{minipage}{0.4\textwidth}
    \animategraphics[autoplay,loop,width=0.8\textwidth]{5}{fig/gif/kmeanAnim-}{0}{12} %% Insert gif as png list
  \end{minipage}
  \vspace{-1cm}
  \begin{itemize}
  \item \textbf{\textcolor{orange}{Limitations:}}
    \begin{itemize}
      \normalsize
    \item Le nombre de clusters se définit ``\textit{à la main}''
    \item Le résultat est très dépendant de l'initialisation
    \item Le \textbf{K-Means} ne fonctionne pas avec les variables catégorielles (non ordonnées)
    \end{itemize}
  \end{itemize}
\end{frame}

%\begin{frame}{3.1 K-means}
%  \begin{itemize}
%  \item Sépare les données en $K$ clusters $C$ d'égale variance (dispersion)
%  \item Soit les \textit{centroïdes}, la moyenne des échantillons dans chaque cluster
%  \item \textit{K-means} modifie la position des \textit{centroïdes} jusqu'à trouver la valeur qui minimise l'écart moyens des échantillons vis-à-vis de son cluster correspondant
%  \item Critère de minimisation: \textbf{inertie}: $I = \displaystyle\sum_{k=0}^{K}\displaystyle\sum_{x \in C_{k}} || x - \mu_{k} ||^{2}$
%  \item La performance du \textit{K-means} est fortement dépendante de son initialisation \textit{(Solution: plusieurs initialisation $\rightarrow$ moyenne)}
%  \item Le \textit{K-means} ne fonctionne pas avec des variables catégorielles (il existe des adaptations). Il est préférable de normaliser les variables.
%  \end{itemize}
%\end{frame}

\begin{frame}{3.1 Clustering: Iris dataset, K-means result}
  \begin{figure}
    \includegraphics[width=0.7\textwidth]{fig/clusteringTrue.png}
    \includegraphics[width=0.7\textwidth]{fig/clusteringKmeans.png}
  \end{figure}
\end{frame}

\begin{frame}{3.1 Mean Shift}
  \begin{itemize}
  \item Cherche les zones de fortes densité en modifiant itérativement la position de \textit{centroïdes}
  \item Des \textit{centroïdes} de rayons $R$ sont aléatoirement initialisés
  \item Ils sont déplacés vers la région de plus haute densité (nombres de points dans le rayon $R$)
  \item On continue jusqu'à maximiser la densité de chaque \textit{centroïde}
  \item Plusieurs \textit{centroïdes} dans une zone: celui avec la plus haute densité est conservé
  \item l'ensemble du dataset est labelisé (plus petite distance)
  \item Le \textit{Mean Shift} détermine le nombre optimal de clusters pour la valeur du rayon choisie (R)
  \end{itemize}
\end{frame}

\begin{frame}{3.1 Clustering: Iris dataset, Mean Shift result}
  \begin{figure}
    \includegraphics[width=0.7\textwidth]{fig/clusteringTrue.png}
    \includegraphics[width=0.7\textwidth]{fig/clusteringMeanShift.png}
  \end{figure}
\end{frame}

\begin{frame}{3.1 Gaussian mixture models}
  \begin{itemize}
  \item \textbf{Hypothèse:} la structure des données est compatible avec un mélange de gaussiènne
  \end{itemize}
  \begin{minipage}{0.48\textwidth}
    \begin{itemize}
    \item On ajuste les paramètres de $N$ gaussiènnes jusqu'à trouver ceux qui \textit{collent} le mieux aux données
    \end{itemize}
  \end{minipage}
  \begin{minipage}{0.48\textwidth}
    \begin{figure}
      \includegraphics[width=0.8\textwidth]{fig/gaussianFct.png}
    \end{figure}
    \tiny
    \vspace{-0.9cm}
    \begin{center}
      depuis \href{https://en.wikipedia.org/wiki/Gaussian_function}{\color{blue}{Wikipedia}}
    \end{center}
  \end{minipage}
  \begin{itemize}
  \item Algorithme d'ajustement: \textbf{espèrance-maximisation}, on cherche à maximiser la fonction de \textit{log-likelihood} (fonction de vraissemblance)
  \item \textit{Intuitivement:} on cherche les paramètres qui rendent la distribution de données observée la plus probable
  \end{itemize}
\end{frame}

\begin{frame}{3.1 Clustering: Iris dataset, GMM result}
  \begin{figure}
    \includegraphics[width=0.7\textwidth]{fig/clusteringTrue.png}
    \includegraphics[width=0.7\textwidth]{fig/clusteringGMM.png}
  \end{figure}
\end{frame}

\begin{frame}{3.2 Analyse en composantes principales}
  \begin{itemize}
  \item \textbf{PCA} (\textit{Composants principal analysis}): Algorithme de réduction de dimensions
  \item Transforme un problème à $n$ variables en un problème à $n'$ variables (avec $n' < n$)
  \item Explore les corrélations entre les variables pour les \textit{regrouper} (pondération): \textbf{Pertes d'informations}
  \item Très utile pour:
    \begin{itemize}
      \normalsize
    \item Pre-stage d'un algorithme de clustering (peut améliorer les performances)
    \item Pour faire de la visualisation en 2D de données à plus haute dimension
    \end{itemize}
  \end{itemize}
\end{frame}

\begin{frame}{3.2 Analyse en composantes principales}
  \begin{itemize}
  \item \textbf{Exemple d'utilisation:} digit dataset:
    \begin{itemize}
      \normalsize
    \item 64 variables $\Rightarrow$ 10 pour initialiser le K-means
    \item 64 $\Rightarrow$ 2 pour la visualisation
    \end{itemize}
  \end{itemize}
  \begin{figure}
    \includegraphics[width=0.6\textwidth]{fig/PCA.png}
  \end{figure}
\end{frame}

\begin{frame}{4. L'apprentissage profond}
  \begin{itemize}
  \item \textbf{Deep Learning} = \textit{Réseaux de Neurones} (avec plus d'une couche cachée)
  \item Conceptualisés dans les années 80 et début 90, l'explosion de la puissance de calcul disponibles a rendu possible leur exploitation
  \item \textbf{Qu'est ce que ça a à voir avec le cerveaux?} Pas grand chose en fait ... à part une analogie avec la structure des neurones
  \end{itemize}
\end{frame}

\begin{frame}{4. L'apprentissage profond}
  \begin{itemize}
  \item Utilisé principalement dans le cadre de l'\textbf{apprentissage supervisé}
    \begin{itemize}
      \normalsize
    \item Image classifiers, Object detection
    \item Speech recognition
    \item Machine traduction (\href{https://www.deepl.com/translator}{\color{blue}{DeepL}})
    \item Voiture autonomes
    \item $\dots$
    \end{itemize}
  \item Données structurées/non-structurées:
    \begin{itemize}
      \normalsize
    \item Les humains sont bons pour interpréter des données non-structurées
    \end{itemize}
  \end{itemize}
\end{frame}

\begin{frame}{4. L'apprentissage profond}
  \begin{figure}
    \includegraphics[width=0.8\textwidth]{fig/dataPerfAlgo.PNG}
  \end{figure}
  \vspace{-1cm}
  \begin{center}
    \footnotesize
    \textcolor{green}{Large NN}, \textcolor{blue}{Medium NN}, \textcolor{brown}{Small NN}, \textcolor{red}{Traditional learning algo}\\
    \vspace{0.1cm}
    \scriptsize
    Depuis \href{https://www.coursera.org/specializations/deep-learning}{\color{blue}{deeplearning.ai coursera}}
  \end{center}
\end{frame}
  
\begin{frame}{4.1 Les réseaux neurones}
  \begin{itemize}
  \item 1 input layer $\Rightarrow$  $L-1$ hidden layer $\Rightarrow$ 1 output layer
  \item $n^{[l]}$ cellules (neurones) pour la couche $l$, $m$ variables (input layer: $n^{[0]}$)
  \item $W^{[l]}, b^{[l]}$: paramètres de la couche $l$ ($W^{[l]} \in \mathbb{R}^{(n^{[l]} \times n^{[l-1]})}$, $b^{[l]} \in \mathbb{R}^{(n^{[l]} \times 1)}$)
  \end{itemize}
  \begin{figure}
    \includegraphics[width=0.6\textwidth]{fig/deepNN.png}
  \end{figure}
\end{frame}

\begin{frame}{4.1 Un neurone $i$ de la couche $l$}
  \begin{figure}
    \includegraphics[trim={2cm 6cm 2cm 1.8cm},clip,width=0.7\textwidth]{fig/neuronEx.jpg}
  \end{figure}
  \begin{itemize}
  \item 2 étapes de calculs: 
    \begin{itemize}
      \normalsize
    \item $z_{i}^{[l]} = \displaystyle\sum_{j=0}^{n^{[l-1]}} w_{ij}^{[l]} \times a_{j}^{[l-1]} + b_{i}^{[l]}$ \hspace{2cm} Où, $a^{[0]} = x$
    \item $a_{i}^{[l]} = f^{[l]}(z_{i}^{[l]})$ \hfill Où, $f^{[l]}(z)$ est la \textbf{fonction d'activation}
    \end{itemize}
  \end{itemize}
\end{frame}

\begin{frame}{4.1 Une couche $l$ de neurones}
  \begin{itemize}
  \item On répète l'opération pour chaque neurone $i$ de la couche $l$
  \item Plus efficace de réfléchir en multiplication de matrices:
  \end{itemize}
  \begin{equation*}
    z^{[l]} = \begin{bmatrix*} z_{1}^{[l]} \\ z_{2}^{[l]} \\ \vdots \\ a_{n^{[l]}}^{[l]} \end{bmatrix*} = \begin{bmatrix*} w_{11}^{[l]} & w_{12}^{[l]} & \dots & w_{1n^{[l-1]}}^{[l]} \\ w_{21}^{[l]} & w_{22}^{[l]} & \dots & w_{2n^{[l-1]}}^{[l]} \\ & & \ddots & \\ w_{n^{[l]}1}^{[l]} & w_{n^{[l]}2}^{[l]} & \dots & w_{n^{[l]}n^{[l-1]}}^{[l]} \end{bmatrix*} \begin{bmatrix*} a_{1}^{[l-1]} \\ a_{2}^{[l-1]} \\ \vdots \\ a_{n^{[l-1]}}^{[l-1]} \end{bmatrix*} + \begin{bmatrix*} b_{1}^{[l]} \\ b_{2}^{[l]} \\ \vdots \\ b_{n^{[l]}}^{[l]} \end{bmatrix*}
  \end{equation*}
  \begin{itemize}
  \item Que l'on réécrira: \boldmath $z^{[l]} = W^{[l]} a^{[l-1]} + b^{[l]}$
  \item Puis: \boldmath $a^{[l]} = f^{[l]}(z^{[l]})$
  \end{itemize}
\end{frame}

\begin{frame}{4.1 Tous les exemples à la fois}
  \begin{itemize}
  \item On a vu le cas avec un exemple, mais si on veux faire les $m$ exemples en une fois? \texttt{for-loop}? $\Rightarrow$ \textbf{Matrices}!
  \item On vectorise: $X \in \mathbb{R}^{n \times m}$, plus généralement: $A^{[l]} (Z^{[l]}) \in \mathbb{R}^{n^{[l]} \times m }$
  \end{itemize}
  \begin{equation*}
    A^{[l]} = \begin{bmatrix*} a_{1}^{[l](1)} & a_{1}^{[l](2)} & \dots & a_{1}^{[l](m)}\\ a_{2}^{[l](1)} & a_{2}^{[l](2)} & \dots & a_{2}^{[l](m)}\\ & & \ddots & \\ a_{n^{[l]}}^{[l](1)} & a_{n^{[l]}}^{[l](2)} & \dots & a_{n^{[l]}}^{[l](m)}\\  \end{bmatrix*}
  \end{equation*}
  \begin{itemize}
  \item Les étapes de calculs s'écrivent: \boldmath $Z^{[l]} = W^{[l]} A^{[l-1]} + b^{[l]}$
  \item Puis: \boldmath $A^{[l]} = f^{[l]}(Z^{[l]})$
  \end{itemize}
\end{frame}

\begin{frame}{4.1 Les 4 principales fonctions d'activations}
  \begin{minipage}{0.48\textwidth}
    \begin{itemize}
    \item \textbf{Logistique:}
    \end{itemize}
    \vspace{-0.5cm}
    \begin{figure}
      \includegraphics[width=0.8\textwidth,height=0.3\textheight]{fig/logisticFct.png}
    \end{figure}
  \end{minipage}
  \begin{minipage}{0.48\textwidth}
    \begin{itemize}
    \item \textbf{Tangente-hyperbolique:}
    \end{itemize}
    \vspace{-0.5cm}
    \begin{figure}
      \includegraphics[width=0.8\textwidth,height=0.3\textheight]{fig/tanhFct.png}
    \end{figure}
  \end{minipage}
  \vfill
  \begin{minipage}{0.48\textwidth}
    \begin{itemize}
    \item \textbf{Rectified Linear Unit:}
    \end{itemize}
    \vspace{-0.2cm}
    \begin{figure}
      \includegraphics[width=0.8\textwidth,height=0.3\textheight]{fig/reluFct.jpeg}
    \end{figure}
  \end{minipage}
  \begin{minipage}{0.48\textwidth}
    \begin{itemize}
    \item \textbf{Leaky ReLU:}
    \end{itemize}
    \vspace{-0.5cm}
    \begin{figure}
      \includegraphics[width=0.8\textwidth,height=0.3\textheight]{fig/leakyReluFct.png}
    \end{figure}
  \end{minipage}
  \begin{itemize}
  \item Si $\hat{y} \in \{0,1\}$: \textbf{Logistique}. Les autres neurones: \textbf{ReLU}
  \end{itemize}
\end{frame}

\begin{frame}{4.1 Forward propagation}
  \begin{minipage}{0.62\textwidth}
    \begin{itemize}
    \item \textcolor{red}{Input \boldmath $X$ \unboldmath}: 3 variables ($n^{[0]} = 3$)
    \item \textcolor{blue}{2 couches cachées} ($n^{[1]} = n^{[2]} = 4$): \textbf{ReLU}
    \item \textcolor{forestGreen}{Output}: $n^{[3]} = 1$: \textbf{Logistique}
    \end{itemize}
  \end{minipage}
  \begin{minipage}{0.36\textwidth}
    \begin{figure}
      \includegraphics[width=1.0\textwidth]{fig/threelayerNN.jpg}
      \put(-120, -5){\vector(1, 0){120}}
      \put(-90, -10){\tiny \textbf{Forward propagation}}
    \end{figure}
  \end{minipage}
  \vfill
  \begin{minipage}{0.59\textwidth}
    \begin{equation*}
      \begin{matrix*}[l]
        X & \in & \mathbb{R}^{3 \times m} & & Y & \in & \mathbb{R}^{1 \times m}\\
        W^{[1]} & \in & \mathbb{R}^{4 \times 3} & & b^{[1]} & \in & \mathbb{R}^{4 \times 1}\\
        Z^{[1]} & \in & \mathbb{R}^{4 \times m} & & A^{[1]} & \in & \mathbb{R}^{4 \times m}\\
        W^{[2]} & \in & \mathbb{R}^{4 \times 4} & & b^{[2]} & \in & \mathbb{R}^{4 \times 1}\\
        Z^{[2]} & \in & \mathbb{R}^{4 \times m} & & A^{[2]} & \in & \mathbb{R}^{4 \times m}\\
        W^{[3]} & \in & \mathbb{R}^{1 \times 4} & & b^{[3]} & \in & \mathbb{R}^{1 \times 1}\\
        Z^{[3]} & \in & \mathbb{R}^{1 \times m} & & A^{[3]} & \in & \mathbb{R}^{1 \times m}\\
      \end{matrix*}
    \end{equation*}
  \end{minipage}
  \begin{minipage}{0.39\textwidth}
    \begin{equation*}
      \begin{matrix*}[l]
        Z^{[1]} = W^{[1]} X + b^{[1]}\\
        A^{[1]} = f^{[1]}(Z^{[1]}) = relu(Z^{[1]})\\
        Z^{[2]} = W^{[2]} A^{[1]} + b^{[2]}\\
        A^{[2]} = f^{[2]}(Z^{[2]}) = relu(Z^{[2]})\\
        Z^{[3]} = W^{[3]} A^{[2]} + b^{[3]}\\
        \hat{Y} = A^{[3]} = f^{[3]}(Z^{[3]}) = \sigma(Z^{[3]})\\
      \end{matrix*}
    \end{equation*}
  \end{minipage}
\end{frame}

\begin{frame}{4.1 Backward propagation}
  \begin{itemize}
  \item C'est l'algorithme d'\textit{apprentissage} des réseaux de neurones
  \item On définit la fonction de coût:
  \end{itemize} 
  \begin{equation*}
    J(W^{[1]},\dots,W^{[L]},b^{[1]},\dots,b^{[L]}) = \frac{1}{m}\displaystyle\sum_{i=1}^{m}L(\hat{y}^{(i)},y^{(i)})
  \end{equation*}
  \vspace{-0.5cm}
  \begin{itemize}
  \item On utilise la \textbf{descente de gradient}:
  \end{itemize}
  \begin{equation*}
    \begin{matrix*}[l]
      \text{Répéter:} & \{ & \\ & & \text{Calculez: } \hat{Y} \\ & & dW^{[L]} := \frac{dJ}{dW^{[L]}}, ~~ db^{[L]} := \frac{dJ}{db^{[L]}}\\ & & W^{[L]} := W^{[L]} - \alpha dW^{[L]} \\ & & b^{[L]} := b^{[L]} - \alpha db^{[L]} \\ & & \vdots \\ & \} &
    \end{matrix*}
  \end{equation*}
\end{frame}


\begin{frame}{4.1 Backward propagation}
  \begin{itemize}
  \item On va propager l'erreur $\hat{Y} - Y$ pour modifier les valeurs des paramètres $W^{[l]}$ et $b^{[l]}$ du réseau de neurones
  \end{itemize}
  \begin{minipage}{0.36\textwidth}
    \begin{figure}
      \includegraphics[width=1.0\textwidth]{fig/threelayerNN.jpg}
      \put(0, -5){\vector(-1, 0){120}}
      \put(-90, -10){\tiny \textbf{Backward propagation}}
    \end{figure}
  \end{minipage}
  \begin{minipage}{0.62\textwidth}
    \begin{equation*}
      \begin{matrix*}[l]
        dZ^{[3]} = A^{[3]} - Y\\
        dW^{[3]} = \frac{1}{m} dZ^{[3]} A^{[2]}{}^{T}\\
        db^{[3]} = \frac{1}{m}\sum dZ^{[3]} \\
        dZ^{[2]} = W^{[3]}{}^{T} dZ^{[3]} * relu'(Z^{[2]})\\
        dW^{[2]} = \frac{1}{m} dZ^{[2]} A^{[1]}{}^{T}\\
        db^{[2]} = \frac{1}{m}\sum dZ^{[2]} \\
        dZ^{[1]} = W^{[2]}{}^{T} dZ^{[2]} * relu'(Z^{[1]})\\
        dW^{[1]} = \frac{1}{m} dZ^{[1]} X^{T}\\
        db^{[1]} = \frac{1}{m}\sum dZ^{[1]} \\
      \end{matrix*}
    \end{equation*}
  \end{minipage}
  \vspace{0.5cm}
  \begin{itemize}
  \item Pour que l'apprentissage fonctionne: \textbf{Initialisation aléatoire} $W,b$
  \end{itemize}
\end{frame}

\begin{frame}{4.2 Les bonnes pratiques: préparer les donneés}
  \begin{itemize}
  \item On sépare les données en trois sous-échantillons:
    \begin{itemize}
      \normalsize
    \item \textbf{Train:} (60$\%$) échantillon d'entrainement avec lequel on applique la \textit{forward-backward propagation}
    \item \textbf{Validation:} (20$\%$) échantillon qui nous permet de mesurer les performances du modèle pour différentes valeurs d'hyperparamètres et différentes architectures
    \item \textbf{Test:} (20$\%$) échantillon de test qui donne la performance du modèle final
    \end{itemize}
  \item Il est important de séparer les données en différents sous-échantillons de test pour être sûr que le modèle de généralise bien
  \item S'assurer que les sous-échantillons proviennent de la même source et soient représentatifs
  \end{itemize}
\end{frame}

\begin{frame}{4.2 Les bonnes pratiques: biais et variance}
  \begin{itemize}
  \item Si $J_{train}(W,b) >> J_{valid}(W,b)$: problème de variance: \textbf{Overfitting}
  \item Si $J_{train}(W,b) \approx J_{valid}(W,b) >> 0$: problème de biais: \textbf{Underfitting}
  \item Comment diagnostiquer? deux valeurs à regarder:
    \begin{itemize}
      \normalsize
    \item Erreur de l'échantillon d'entrainement: $t_{err}$
    \item Erreur de l'échantillon de validation: $v_{err}$
    \end{itemize}
  \item On estime le cas idéal (Bayes ou opérateur humain): $\approx 0\%$
  \end{itemize}
  \vspace{0.5cm}
  \begin{minipage}{.39\textwidth}
    \begin{figure}
      \includegraphics[trim={0 54 0 0},clip,width=\textwidth]{fig/biasVsVariance}
    \end{figure}
    \vspace{-1cm}
    \begin{center}
      \scriptsize
      \href{https://www.coursera.org/learn/machine-learning}{\color{blue}{[Coursera]}}
    \end{center}
  \end{minipage}
  \begin{minipage}{.59\textwidth}
    \begin{table}
      \begin{tabular}{l|l|l}
        $t_{err}$ & $v_{err}$ & problème \\
        \hline
        1$\%$ & 11$\%$ & haute variance \\
        15$\%$ & 16$\%$ & haut biais \\
        15$\%$ & 30$\%$ & haut biais \& haute variance \\
        0.5$\%$ & 1$\%$ & bas biais \& basse variance \\
      \end{tabular}
    \end{table}
  \end{minipage}
\end{frame}

\begin{frame}{4.2 Les bonnes pratiques: courbes d'apprentissage}
  \begin{itemize}
  \item Entrainer un algo en augmentant le nombre d'exemples et monitorer l'évolution de l'erreur:
  \end{itemize}
  \begin{figure}
    \includegraphics[trim = {0 0 0 32}, clip, width=0.44\textwidth]{fig/learningCurves1.png}
    \includegraphics[trim = {0 0 0 30}, clip, width=0.48\textwidth]{fig/learningCurves2.png}
  \end{figure}
  \vspace{-0.5cm}
  \begin{center}
    \scriptsize
    Biais et variance \href{https://www.coursera.org/learn/machine-learning}{\color{blue}{[Coursera]}}
  \end{center}  
\end{frame}

\begin{frame}{4.2 Les bonnes pratiques: Que faire?}
  \begin{table}
    \footnotesize
    \begin{tabular}{ccccl}
      \multicolumn{3}{l}{\textcolor{orange}{\textbf{$\bullet$ Haut Biais?}}} & \boldmath $\rightarrow$ \textbf{Oui} $\rightarrow$ \unboldmath & Réseau plus profond \\
      \multicolumn{3}{l}{(Performance entrainement)} & & Entrainer plus longtemps \\
      & & & & Changer architecture du NN \\
      & & \boldmath $\downarrow$ \unboldmath & & \multicolumn{1}{c}{\boldmath $\downarrow$ \boldmath} \\
      & & \textbf{Non} & & \multicolumn{1}{c}{\textit{\textcolor{orange}{Recommencer!}} \boldmath $\rightarrow$ \unboldmath}\\
      & & \boldmath $\downarrow$ \unboldmath & & \\
      & & & & \\
      \multicolumn{3}{l}{\textcolor{orange}{\textbf{$\bullet$ Haute Variance?}}} & \boldmath $\rightarrow$ \textbf{Oui} $\rightarrow$ \unboldmath & Plus de données \\
      \multicolumn{3}{l}{(Performance validation)} & & Regularisation \\
      & & & & Changer architecture du NN \\
      & & \boldmath $\downarrow$ \unboldmath & & \multicolumn{1}{c}{\boldmath $\downarrow$ \boldmath} \\
      & & \textbf{Non} & & \multicolumn{1}{c}{\textit{\textcolor{orange}{Recommencer!}} \boldmath $\rightarrow$ \unboldmath}\\
      & & \boldmath $\downarrow$ \unboldmath & & \\
      & & & & \\
      & & \textcolor{orange}{\textbf{OK!}} & & \\      
    \end{tabular}
  \end{table}
\end{frame}

\begin{frame}{4.3 Les différentes architectures de NN}
  \begin{itemize}
  \item Différentes architectures pour différents usages:
  \end{itemize}
  \vspace{0.5cm}
  \footnotesize
  \begin{minipage}{.3\textwidth}
    \begin{center}
      \textbf{\textcolor{orange}{Std NN}}
    \end{center}
    \begin{figure}
      \includegraphics[width=0.9\textwidth,height=0.2\textheight]{fig/threelayerNN.jpg}
    \end{figure}
    Regression\\
    Classification\\
  \end{minipage} 
  \begin{minipage}{.3\textwidth}
    \begin{center}
      \textbf{\textcolor{orange}{Convolutional NN}}
    \end{center}
    \begin{figure}
      \includegraphics[width=0.9\textwidth,height=0.2\textheight]{fig/CNN.png}
    \end{figure}
    Image classifier\\
    Object detection\\
  \end{minipage}  
  \begin{minipage}{.3\textwidth}
    \begin{center}
      \textbf{\textcolor{orange}{Recurent NN}}
    \end{center}
    \begin{figure}
      \includegraphics[width=0.9\textwidth,height=0.2\textheight]{fig/RNN.png}
    \end{figure}
    Audio recognition\\
    Time Series analysis\\
  \end{minipage}
\end{frame}



\setbeamertemplate{background}{\pgfuseimage{bkg}}
\begin{frame}
  \begin{center}
    \huge
    \vspace{5cm}
    \textbf{\textcolor{orange}{Merci!}}\\
    \textbf{Des questions?}\\
    \vspace{1cm}
    \normalsize
    \textit{Léo Beaucourt: lbeaucourt@agaetis.fr}
  \end{center}
\end{frame}

\end{document}
